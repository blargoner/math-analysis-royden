% Notes and exercises from Real Analysis by Royden
% By John Peloquin
\documentclass[letterpaper,12pt]{article}
\usepackage{amsmath,amssymb,amsthm,enumitem,fourier}

\newcommand{\Q}{\mathbb{Q}}
\newcommand{\R}{\mathbb{R}}
\newcommand{\Gd}{G_{\delta}}
\newcommand{\Fs}{F_{\sigma}}

\newcommand{\union}{\cup}
\newcommand{\sect}{\cap}
\newcommand{\bigunion}{\bigcup}
\newcommand{\bigsect}{\bigcap}

\newcommand{\len}{l}
\newcommand{\mo}{m^*}

\newcommand{\comp}[1]{\widetilde{#1}}
\newcommand{\closure}[1]{\overline{#1}}

% Theorems
\theoremstyle{plain}
\newtheorem*{lem}{Lemma}
\newtheorem*{prop}{Proposition}

\theoremstyle{definition}
\newtheorem*{exer}{Exercise}

\theoremstyle{remark}
\newtheorem*{rmk}{Remark}

% Meta
\title{Notes and exercises from \textit{Real Analysis}}
\author{John Peloquin}
\date{}

\begin{document}
\maketitle

\section*{Introduction}
This document contains notes and exercises from~\cite{royden}.

\section*{Chapter~3}
\subsection*{Section~2}
\begin{prop}[5]
Given any \(A\subseteq\R\) and any \(\epsilon>0\), there is an open set~\(O\) such that \(A\subseteq O\) and \(\mo O\le\mo A+\epsilon\). There is a \(G\in\Gd\) such that \(A\subseteq G\) and \(\mo A=\mo G\).
\end{prop}
\begin{proof}
If \(\mo A=\infty\), take \(O=G=\R\). If \(\mo A<\infty\), there is a countable set~\(\{I_n\}\) of open intervals with \(A\subseteq\bigunion I_n\) and \(\sum\len(I_n)<\mo A+\epsilon\). Let \(O=\bigunion I_n\). Then \(O\)~is open (Proposition~2.7), \(A\subseteq O\), and
\[\mo O\le\sum\mo I_n=\sum\len(I_n)<\mo A+\epsilon\]
by countable subadditivity (Proposition~2) and the outer measure of intervals (Proposition~1).

For each~\(n\ge 1\), let \(O_n\)~be open with \(A\subseteq O_n\) and \(\mo O_n\le\mo A+1/n\). Let \(G=\bigsect O_n\). Then \(G\in\Gd\) and \(A\subseteq G\). By monotonicity, \(\mo A\le\mo G\) and
\[\mo G\le\mo O_n\le\mo A+1/n\]
for all \(n\ge 1\), so \(\mo A=\mo G\).
\end{proof}

\begin{exer}[5]
Let \(A=\Q\sect(0,1)\), and let \(\{I_n\}\)~be a finite set of open intervals with \(A\subseteq\bigunion I_n\). Then \(\sum\len(I_n)\ge 1\).
\end{exer}
\begin{proof}
By density of~\(\Q\) in~\(\R\) (Corollary~2.4), \([0,1]=\closure{A}\subseteq\closure{\bigunion I_n}=\bigunion\closure{I_n}\), so
\[1=\len[0,1]=\mo[0,1]\le\mo\bigunion\closure{I_n}\le\sum\mo\closure{I_n}=\sum\len(I_n)\qedhere\]
\end{proof}

\begin{exer}[7]
If \(E\subseteq\R\) and \(y\in\R\), then \(\mo(E+y)=\mo E\).
\end{exer}
\begin{proof}
First, if \(E=(a,b)\) with \(-\infty\le a<b\le\infty\), then \(E+y=(a+y,b+y)\), so
\[\len(E+y)=b-a=\len(E)\]
Now if \(E\)~is arbitrary and \(\{I_n\}\)~is a countable set of open intervals with \(E\subseteq\bigunion I_n\), then \(\{I_n+y\}\)~is a countable set of open intervals with \(E+y\subseteq\bigunion(I_n+y)\) and \(\sum\len(I_n+y)=\sum\len(I_n)\) by the above. Therefore \(\mo(E+y)\le\mo E\). Conversely,
\[\mo E=\mo((E+y)-y)\le\mo(E+y)\]
so \(\mo(E+y)=\mo E\).
\end{proof}

\begin{exer}[8]
If \(\mo A=0\), then \(\mo(A\union B)=\mo B\).
\end{exer}
\begin{proof}
We have \(\mo B\le\mo(A\union B)\le\mo A+\mo B=\mo B\).
\end{proof}

\subsection*{Section~3}
We provide an alternative proof of Lemma~7.
\begin{lem}[7]
If \(E_1,E_2\subseteq\R\) are measurable, so is~\(E_1\union E_2\).
\end{lem}
\begin{proof}
Let \(A\subseteq\R\). We first claim\footnote{This is an inclusion-exclusion inequality for outer measure.}
\[\mo(A\sect[E_1\union E_2])+\mo(A\sect E_1\sect E_2)\le\mo(A\sect E_1)+\mo(A\sect E_2)\tag{1}\]
Indeed, since \(A\sect(E_1\union E_2)=(A\sect E_1\sect E_2)\union(A\sect E_1\sect\comp{E_2})\union(A\sect\comp{E_1}\sect E_2)\),
\[\mo(A\sect[E_1\union E_2])\le\mo(A\sect E_1\sect E_2)+\mo(A\sect E_1\sect\comp{E_2})+\mo(A\sect\comp{E_1}\sect E_2)\]
by subadditivity, so
\begin{align*}
\mo(A\sect[E_1\union E_2])+\mo(A\sect E_1\sect E_2)&\le\mo([A\sect E_1]\sect E_2)+\mo([A\sect E_1]\sect\comp{E_2})\\
	&\ +\mo([A\sect E_2]\sect E_1)+\mo([A\sect E_2]\sect\comp{E_1})\\
	&=\mo(A\sect E_1)+\mo(A\sect E_2)
\end{align*}
by measurability of \(E_1\)~and~\(E_2\), establishing~(1).

We now claim \(\mo(A\sect[E_1\union E_2])+\mo(A\sect[\comp{E_1\union E_2}])\le\mo A\). If \(\mo A=\infty\), there is nothing to prove, so we assume \(\mo A<\infty\). We have
\begin{align*}
\mo(A\sect[E_1\union E_2])+\mo(A\sect\comp{E_1}\sect\comp{E_2})+\mo A&\le\mo(A\sect[E_1\union E_2])+\mo(A\sect E_1\sect E_2)\\
	&\ +\mo(A\sect[\comp{E_1}\union\comp{E_2}])+\mo(A\sect\comp{E_1}\sect\comp{E_2})\\
	&\le\mo(A\sect E_1)+\mo(A\sect E_2)\\
	&\ +\mo(A\sect\comp{E_1})+\mo(A\sect\comp{E_2})\\
	&=\mo A+\mo A
\end{align*}
where the first inequality follows from \(A=(A\sect E_1\sect E_2)\union(A\sect[\comp{E_1}\union\comp{E_2}])\) and subadditivity, the second inequality follows from~(1), and the equality follows from measurability of \(E_1\)~and~\(E_2\). The claim now follows since \(\mo A<\infty\). Since \(A\)~was arbitrary, \(E_1\union E_2\)~is measurable.
\end{proof}

% References
\begin{thebibliography}{0}
\bibitem{royden} Royden, H.~L. \textit{Real Analysis}, 3rd~ed. Prentice Hall, 1988.
\end{thebibliography}
\end{document}
