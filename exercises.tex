% Notes and exercises from Real Analysis by Royden
% By John Peloquin
\documentclass[letterpaper,12pt]{article}
\usepackage{amsmath,amssymb,amsthm,enumitem,fourier}

\newcommand{\R}{\mathbb{R}}
\newcommand{\Gd}{G_{\delta}}
\newcommand{\Fs}{F_{\sigma}}

\newcommand{\union}{\cup}
\newcommand{\sect}{\cap}
\newcommand{\bigunion}{\bigcup}
\newcommand{\bigsect}{\bigcap}

\newcommand{\len}{l}
\newcommand{\mo}{m^*}

% Theorems
\theoremstyle{plain}
\newtheorem*{prop}{Proposition}

\theoremstyle{definition}
\newtheorem*{exer}{Exercise}

\theoremstyle{remark}
\newtheorem*{rmk}{Remark}

% Meta
\title{Notes and exercises from \textit{Real Analysis}}
\author{John Peloquin}
\date{}

\begin{document}
\maketitle

\section*{Introduction}
This document contains notes and exercises from~\cite{royden}.

\section*{Chapter~3}
\begin{prop}[5]
Given any \(A\subseteq\R\) and any \(\epsilon>0\), there is an open set~\(O\) such that \(A\subseteq O\) and \(\mo O\le\mo A+\epsilon\). There is a \(G\in\Gd\) such that \(A\subseteq G\) and \(\mo A=\mo G\).
\end{prop}
\begin{proof}
If \(\mo A=\infty\), take \(O=G=\R\). If \(\mo A<\infty\), there is a countable set~\(\{I_n\}\) of open intervals with \(A\subseteq\bigunion I_n\) and \(\sum\len(I_n)<\mo A+\epsilon\). Let \(O=\bigunion I_n\). Then \(O\)~is open (Proposition~2.7), \(A\subseteq O\), and
\[\mo O\le\sum\mo I_n=\sum\len(I_n)<\mo A+\epsilon\]
by countable subadditivity (Proposition~2) and the outer measure of intervals (Proposition~1).

For each~\(n\ge 1\), let \(O_n\)~be open with \(A\subseteq O_n\) and \(\mo O_n\le\mo A+1/n\). Let \(G=\bigsect O_n\). Then \(G\in\Gd\) and \(A\subseteq G\). By monotonicity, \(\mo A\le\mo G\) and
\[\mo G\le\mo O_n\le\mo A+1/n\]
for all \(n\ge 1\), so \(\mo A=\mo G\).
\end{proof}

% References
\begin{thebibliography}{0}
\bibitem{royden} Royden, H.~L. \textit{Real Analysis}, 3rd~ed. Prentice Hall, 1988.
\end{thebibliography}
\end{document}
